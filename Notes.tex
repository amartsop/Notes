\input{preamble.tex}

\begin{document}

%%%%%%%%%%%%%%%%%%%% Sections Format %%%%%%%%%%%%%%%%%%%%
\setlength\parindent{0pt}

%%%%%%%%%%%%%%%%% Main report %%%%%%%%%%%%%%%%%%%%

\clearpage
\setcounter{page}{1}

\chapter{Rigid body: Kinematics \& Dynamics}

\section{Kinematics}

Let $P_{i}$ represent an arbitrary point on the rigid body 'i' that is 
shown in Figure \ref{fig:rigid_body} and $C_{i}$ the center the body's center of 
mass. Furthermore, the frame $f_{i}$ is rigidly attached to the body (translates
and rotates with it), while the $F$ frame is the inertial frame of reference.

\begin{figure}[h]
    \centering\includegraphics[scale=0.2]{Images/rigid_body_diagram.png}
    \caption{Rigid Body}
    \label{fig:rigid_body}
\end{figure}

The center of mass of the body can be defined as
\[
    \pvec{r}{OC_{i}}{F}{F} = \frac{1}{m_{i}}\int_{m_i} \pvec{r}{OP_{i}}{F}{F} \  dm_{i}
    \quad \text{where} \quad m_{i} = \int_{m_i} dm_{i}.
\]

\subsection{Position}
The position of the arbitrary point $P_{i}$ with respect to the inertial frame 
is defined as
\begin{equation}    
    \pvec{r}{OP_{i}}{F}{F} = \pvec{r}{OC_{i}}{F}{F} + \rot{R}{f_{i}}{F} \  \pvec{r}{C_{i}P_{i}}{f_i}{f_i},
    \label{eq:position_rigid_body}
\end{equation}

where $\rot{R}{f_{i}}{F}$ is the rotation matrix of body frame $f_{i}$ with respect to 
the inertial frame $F$.

\subsection{Velocity}
The velocity of the the arbitrary point $P_{i}$ with respect to the inertial frame 
is defined as

\[
    \pvec{\dot{r}}{OP_{i}}{F}{F} = \pvec{\dot{r}}{OC_{i}}{F}{F} + 
    \frac{d}{dt}(\rot{R}{f_{i}}{F} \  \pvec{r}{C_{i}P_{i}}{f_i}{f_i}),
\]

\begin{equation}
    \pvec{\dot{r}}{OP_{i}}{F}{F} = \pvec{\dot{r}}{OC_{i}}{F}{F} + 
    \rot{\dot{R}}{f_{i}}{F} \  \pvec{r}{C_{i}P_{i}}{f_i}{f_i} + 
    \rot{R}{f_{i}}{F} \  \pvec{\dot{r}}{C_{i}P_{i}}{f_i}{f_i}.
    \label{eq:velocity_general}
\end{equation}

We know that for a rigid body the distance between to points remains constant 
meaning that $\pvec{\dot{r}}{C_{i}P_{i}}{f_i}{f_i} = 0$. The time derivative of 
the rotation matrix can be defined as 
\begin{equation}
    \rot{\dot{R}}{f_{i}}{F} = S(\pvec{\omega}{f_{i}}{F}{F}) \ \rot{R}{f_{i}}{F},
    \label{eq:rot_mat_derivative}
\end{equation}

where, $\pvec{\omega}{f_{i}}{F}{F}$ the rotational velocity of $f_{i}$ frame 
with resepct to $F$ frame and $S(\vecd{a})$ skew-symetrix matrix defined as 
\[
    S(\vecd{a}) = \begin{bmatrix}
        0 && -a_{3} && a_{2} \\
        a_{3} && 0 && -a_{1} \\ 
        -a_{2} && a_{1} && 0 \\ 
    \end{bmatrix}, \quad \text{where} \quad \vecd{a} = \begin{bmatrix}
        a_{1} \\ a_{2} \\ a_{3}
    \end{bmatrix}.
\]

Given the above equation \eqref{eq:velocity_general} becomes
\begin{align*}
    \pvec{\dot{r}}{OP_{i}}{F}{F} = \pvec{\dot{r}}{OC_{i}}{F}{F} + 
    S(\pvec{\omega}{f_{i}}{F}{F}) \rot{R}{f_{i}}{F} \ \pvec{r}{C_{i}P_{i}}{f_i}{f_i} \\
    & = \pvec{\dot{r}}{OC_{i}}{F}{F} - S(\rot{R}{f_{i}}{F} \ \pvec{r}{C_{i}P_{i}}{f_i}{f_i}) \ \pvec{\omega}{f_{i}}{F}{F} \\ 
    & = \pvec{\dot{r}}{OP_{i}}{F}{F} = \pvec{\dot{r}}{OC_{i}}{F}{F} -
    \rot{R}{f_{i}}{F} \ S( \ \pvec{r}{C_{i}P_{i}}{f_i}{f_i}) \ (\rot{R}{f_{i}}{F})^{T} \ \pvec{\omega}{f_{i}}{F}{F}.  
\end{align*}

or
\begin{equation}
    \pvec{\dot{r}}{OP_{i}}{F}{F} = \pvec{\dot{r}}{OC_{i}}{F}{F} - 
    \rot{R}{f_{i}}{F} \ S( \ \pvec{r}{C_{i}P_{i}}{f_i}{f_i})\ \pvec{\omega}{f_{i}}{F}{f_i}.
    \label{eq:velocity_rigid_body}
\end{equation}

It can be proven that the rotational velocity of the body frame with respect to
the inertial frame (expressed in the body frame) can be written as:

\begin{equation}
    \pvec{\omega}{f_{i}}{F}{f_i} = G(\vecd{\theta_{i}})\dot{\theta_{i}} = 
    G_{i}\dot{\theta_{i}}, \quad G_{i} = G(\vecd{\theta_{i}}),  
    \label{eq:angular_velocity_expression}
\end{equation}

where $\theta_{i}$ is the vector with the parameters that describe the orientation 
of the body frame with respect to the inertial frame (euler angles, euler parameters,
rodrigues parameters, etc.).

If we define the generalized rigid body parameters as 
\begin{equation}
    \vecd{q}_{r_{i}} = \begin{bmatrix}
        \pvec{\dot{r}}{OC_{i}}{F}{F} \\  \vecd{\theta_{i}}
    \end{bmatrix}
\end{equation}
    
    

\end{document}
